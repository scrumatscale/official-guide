\documentclass[12pt,a4paper,parskip=full]{scrartcl}

\usepackage{bbding}
\usepackage{pifont}
\usepackage{wasysym}
\usepackage[margin=1in]{geometry}
\geometry{letterpaper}
\usepackage{xcolor}
\definecolor{red}{HTML}{cc0000}
\definecolor{gray}{HTML}{666666}
\usepackage{sectsty}
\sectionfont{\color{red}}
\subsectionfont{\color{red}}
\usepackage{graphicx}
\usepackage{hyperref}
\usepackage{amssymb}
\usepackage[style=footnote-dw]{biblatex}
\bibliography{S@SGuideBib}
\setlength\bibitemsep{0.5\baselineskip}

\usepackage{enumitem}
\setitemize{noitemsep}
% \setlist{noitemsep, topsep=-5pt}
% \setlength\itemsep{-0.10em}

\renewcommand{\labelitemi}{$\cdot$}
\renewcommand{\labelitemii}{$\cdot$}
\makeatletter
\let\latexl@section\l@section
\def\l@section#1#2{\begingroup\let\numberline\@gobble\latexl@section{#1}{#2}\endgroup}
\makeatother

\usepackage[T1]{fontenc}
\fontfamily{verdana}

\usepackage{scrlayer-scrpage}{}
\makeatletter
\renewcommand{\@seccntformat}[1]{}
\makeatother

\setlength\parindent{0pt}{}

\title{\Huge{\color{red}\textbf{The Scrum@Scale
\textsuperscript{\copyright}
Guide}}}
\subtitle{\color{gray}The Definitive Guide to Scrum@Scale:\\ Scaling that
Works}
% \author{}
\date{}


\begin{document}

%\tableofcontents
%\newpage

\section{Purpose of the Scrum@Scale Guide}
Scrum, as originally outlined in the Scrum Guide, is a framework for
developing, delivering, and sustaining complex products by a single team.
Since its inception, its usage has extended to the creation of products,
processes, services, and systems that require the efforts of multiple
teams. Scrum@Scale was created to efficiently coordinate this new ecosystem
of teams in a way that optimizes the overall strategy of the organization.
It achieves this goal through setting up a ``minimum viable bureaucracy''
via a scale-free architecture, which naturally extends the way a single
Scrum team functions across the organization.

This guide contains the definitions of the components that make up the
Scrum@Scale framework, including its scaled roles, scaled events, and
enterprise artifacts, as well as the rules that bind them together.

Dr. Jeff Sutherland developed Scrum@Scale based on the fundamental
principles behind Scrum, Complex Adaptive Systems theory, game theory, and
object-oriented technology. This guide was developed with the input of many
experienced Scrum practitioners based on the results of their field work.
The goal of this guide is for the reader to be able to implement Scrum@Scale
on their own.

\subsection{Why Scrum@Scale?}
Scrum was designed for a single team to be able to work at its optimal
capacity while maintaining a sustainable pace. In the field, it was found
that as the number of Scrum teams within an organization grew, the 
output (working product) and velocity of those teams began to fall (due to
issues like cross-team dependencies and duplication of work). It became
obvious that a framework for effectively coordinating those teams was
needed in order to achieve linear scalability. Scrum@Scale is designed to
accomplish this goal via its scale-free architecture.

By utilizing a scale-free architecture, the organization is not constrained
to grow in a particular way determined by a set of arbitrary rules; rather
it can grow organically based on its unique needs and at a sustainable pace
of change that can be accepted by the groups of individuals that make up
the organization. The simplicity of the Scrum@Scale model is essential to a scale-free architecture and carefully avoids introducing extra complexity that will cause productivity per team to decrease as more teams are created.

Scrum@Scale is designed to scale across the organization as a whole: all
departments, products and services. It can be applied across multiple
domains in all types of organizations in industry, government, or academia.

\subsection{Definition of Scrum@Scale}
Scrum: A framework within which people can address complex adaptive
problems, while productively and creatively delivering viable products of the
highest possible value.

The Scrum Guide is the minimal feature set that allows inspection and
adaptability via radical transparency to drive innovation, customer satisfaction, performance, and
team happiness.

Scrum@Scale: A framework within which networks of Scrum teams operating
consistently with the Scrum Guide can address complex adaptive problems,
while creatively delivering products of the highest possible value.

\textbf{NOTE:} These ``products'' may be hardware, software, complex
integrated systems, processes, services, etc., depending upon the domain of
the Scrum teams.

Scrum@Scale is:
\begin{itemize}
\item Lightweight - the minimum viable bureaucracy
\item Simple to understand - consists of only Scrum teams
\item Difficult to master - requires implementing a new operating model
\end{itemize}


Scrum@Scale is a framework for scaling Scrum. It radically simplifies
scaling by using Scrum to scale Scrum. 

In Scrum, care is taken to separate accountability of the ``what'' from the
``how''. The same care is taken in Scrum@Scale so that jurisdiction and
accountability are expressly understood in order to eliminate wasteful
organizational conflict that keep teams from achieving their optimal
productivity.

Scrum@Scale consists of components that allow an organization to
customize their transformational strategy and implementation. It gives them
the ability to target their incrementally prioritized change efforts in the area(s) they deem
most valuable or most in need of change and then progress on to others.

In separating these two jurisdictions, Scrum@Scale contains two cycles: the
Scrum Master Cycle (the ``how'') and the Product Owner Cycle (the
``what''), each touching the other at two points. Taken together, these
cycles produce a powerful framework for coordinating the efforts of
multiple teams along a single path.

\subsection{The Components of the Scrum@Scale\textregistered ~Framework}

\includegraphics[width=1.0\linewidth]{SMPO-Cycle.png}

\subsection{Values-Driven Culture}
Besides separating accountability of the ``what'' and the ``how,''
Scrum@Scale further aims to build healthy organizations by creating a
values-driven culture in an empirical setting. The Scrum values are:
Openness, Courage, Focus, Respect, and Commitment. These values drive
empirical decision making, which depend on the three pillars of
transparency, inspection, and adaptation.

Openness supports transparency into all of the work and processes, without
which there is no ability to inspect them honestly and attempt to adapt
them for the better. Courage refers to taking the bold leaps required to
deliver value quicker in innovative ways.

Focus and Commitment refer to the way we handle our work obligations,
putting customer value delivery as the highest priority. Lastly, all of
this must occur in an environment based on respect for the individuals
doing the work, without whom nothing can be created.

Scrum@Scale helps organizations thrive by supporting a
transformational leadership model which fosters a positive environment for
working at a sustainable pace and putting commitment to deliver
customer-facing value at the forefront of our efforts.

\subsection{Getting Started with Scrum@Scale}
When implementing large networks of teams, it is critical to develop a
scalable \textbf{Reference Model} for a small set of teams. Any
deficiencies in a Scrum implementation will be magnified when multiple
teams are deployed. Many of the initial scaling problems will be organizational 
policies and procedures or development practices that block high performance and frustrate 
teams. 

Therefore, the first challenge is to create a small set of teams that
implements Scrum well.  This is best accomplished by the creation of an 
\textbf{Executive Action Team (EAT)}, which is accountable for the development and 
execution of the transformation strategy.  The EAT must be comprised of individuals 
who are empowered, politically and financially to ensure the existence of the
Reference Model.  This set of teams works through organizational
issues that block agility and creates a Reference Model for Scrum that is
known to work in the organization and can be used as a pattern for scaling
Scrum across the organization.

As the Reference Model of teams accelerates, impediments and bottlenecks
that delay delivery, produce waste, or impede business agility become
apparent. The most effective way to eliminate these problems is to spread
Scrum across the organization so that the entire value stream is optimized.

Scrum@Scale achieves linear scaling in productivity by saturating the
organization with Scrum and distributing velocity and quality organically,
consistent with the organization's specific strategy, product, and services.

\section{Scrum Master Cycle}
\subsection{Team-Level Process}
The \textbf{Team-Level Process} constitutes the first touch point between the Scrum Master and Product Owner Cycles, and is laid out clearly in the Scrum Guide. It
is composed of three artifacts, five events, and three roles. The goals of
the team level process are to:
\begin{itemize}
\item maximize the flow of completed and quality tested work.
\item increase performance of the team over time.
\item operate in a way that is sustainable and enriching for the team.
\item accelerate the customer feedback loop.
\end{itemize}

\subsection{Coordinating the ``How'' - The Scrum of Scrums}
A set of the teams that need to coordinate in order to deliver value to customers comprise a \textbf{``Scrum of Scrums'' (SoS)}. This team  is responsible for a fully integrated set of potentially shippable increments of product at the end of every Sprint from all participating teams. A SoS functions as a Release Team and must be able to directly deliver value to customers. To do so effectively, it needs to be consistent with the Scrum Guide; that is, have its own roles, artifacts, and events:

Roles:

The SoS needs to have all of the skills necessary to deliver a fully integrated potentially shippable Product Increment at the end of every Sprint and to facilitate cross-team coordination where necessary. (It may need experienced architects, QA Leaders, members of the Product Ownership Team, and other operational skill sets.) 
The Scrum Master of the Scrum of Scrums is called the \textbf{Scrum of Scrums Master (SoSM)}.

Events:

The SoSM should facilitate a Backlog Refinement event wherein impediments are identified as ``ready'' to be removed, and the team determines how best to remove them and how they will know when they are ``done.'' These impediment removal backlog items/ artifacts are prioritized for the teams on the same single backlog produced by the Meta Scrum. 
Particular attention should be paid to the SoS Retrospective in which the teams' representatives share any learnings or process improvements that their individual teams have succeeded with, in order to share best practices across the teams within the SoS.  
Since the SoS needs to be responsive in real-time to impediments raised by participating teams, at least one representative (usually the team's Scrum Master) of each of the participating teams need to attend a \textbf{Scaled Daily Scrum (SDS)}. The SDS event mirrors the Daily Scrum in that it optimizes the collaboration and performance of the network of teams. Any person or number of people from participating teams may attend as needed.

Additionally, the SDS:

\begin{itemize}
\item is time-boxed to 15 minutes or less.
\item must be attended by a representative of each team including the Product Owner team.
\item is a forum where team representatives discuss what is going well, what is getting done, and how teams can work together more effectively. Some examples of what might be discussed are:
\begin{itemize}
\item What impediments does my team have that will prevent them from
accomplishing their Sprint Goal (or impact the upcoming release)?
\item Is my team doing anything that will prevent another team from
accomplishing their Sprint Goal (or impact their upcoming release)?
\item Have we discovered any new dependencies between the teams or
discovered a way to resolve an existing dependency?
\item What improvements have we discovered that can be leveraged across teams?
\end{itemize}
\end{itemize}

\subsection{The Scrum of Scrums Master (SoSM)}
The Scrum of Scrums Master (SoSM) is accountable for the release of the
joint teams' effort and must:
\begin{itemize}
\item make progress visible.
\item make an impediment backlog visible to the organization.
\item remove impediments that the teams cannot address themselves.
\item facilitate prioritization of impediments, with particular attention to cross-team
dependencies and the distribution of backlog.
\item improve the efficacy of the Scrum of Scrums.
\item work closely with the Product Owners to deploy a potentially
releasable Product Increment at least every Sprint.
\item coordinate the teams' deployment with the Product Owner's Release
Plans.
\end{itemize}

\subsection{Scaling the SoS}
Depending upon the size of the organization or implementation, more than
one SoS may be needed to deliver a very complex product. In those cases, a
\textbf{Scrum of Scrum of Scrums (SoSoS)} can be created out of multiple
Scrums of Scrums. The SoSoS is an organic pattern of Scrum teams which is
infinitely scalable. Each SoSoS should have SoSoSM's and scaled versions of
each artifact \& event.

Scaling the SoS reduces the number of communication pathways within the
organization so that complexity is encapsulated. The SoSoS interfaces with
a SoS in the exact same manner that a SoS interfaces with a single Scrum
team which allows for linear scalability.

\pagebreak
Sample Diagrams:

\includegraphics[width=1.0\linewidth]{Sos-R2.png}

\textbf{\textsc{note:}} While the Scrum Guide defines the optimal team size as being
3 to 9 people, Harvard research determined that optimal team size is 4.6
people.\footnote{Hackman, J Richard, Leading teams: Setting the stage for
great performances, Harvard Business Press, 2002} Experiments with high
performing Scrum teams have repeatedly shown that 4 or 5 people doing the
work is the optimal size. It is essential to linear scalability that this
pattern be the same for the number of teams in a SoS. Therefore, in the
above and following diagrams, pentagons were chosen to represent a team of
5. These diagrams are meant to be examples only, your organizational
diagram may differ greatly.

\subsection{The Executive Action Team}
The Scrum of Scrums for the entire agile organization is called the
\textbf{Executive Action Team (EAT)}. The leadership team creates an agile bubble
in the organization where the Reference Model operates with 
its own guidelines and procedures that integrates effectively 
with any part of the organization that is not agile. It owns the agile ecosystem, 
implements the Scrum values, and assures that 
Scrum roles are created and supported.

The EAT is the final stop for
impediments that cannot be removed by the SoS's that feed it. Therefore, it
must be comprised of individuals who are empowered, politically and
financially, to remove them. 
The function of the EAT is to coordinate
multiple SoS's (or SoSoS's) and to interface with any non-agile parts 
of the organization. As with any Scrum team, it needs a PO and SM and to meet daily. They must have a transparent backlog as well.

%\pagebreak
Sample Diagram showing an EAT coordinating 5 groupings of 25 teams:

\includegraphics[width=\textwidth,height=\textheight,keepaspectratio]{SoS-EAT.png}

\subsection{The EAT's Backlog \& Responsibilities}
Scrum is an agile operating system that is different from traditional
project management. The entire SM organization reports into the EAT, which
is responsible for implementing this agile operating system by
establishing, maintaining, and enhancing the implementation in the
organization.
The EAT's role is to create an Organizational Transformation Backlog (a
prioritized list of the agile initiatives that need to be accomplished) and
see that it is carried out. For example, if there is a traditional Product
Development Life Cycle in the old organization, a new agile Product
Development Life Cycle needs to be created, implemented, and supported. It
will typically support quality and compliance issues better than the old
method but be implemented in a different way with different rules and
guidelines. The EAT ensures that a Product Owner organization is created and funded
and that this organization is represented on the EAT to support these efforts.

The EAT is accountable for the quality of Scrum within the organization.
Its responsibilities include but are not limited to:
\begin{itemize}
\item creating an agile operating system for the Reference Model as it
scales through the organization, including corporate operational rules,
procedures, and guidelines to enable agility.
\item measuring and improving the quality of Scrum in the organization.
\item building capability within the organization for business agility.
\item creating a center for continuous learning for Scrum professionals.
\item supporting the exploration of new ways of working.
\end{itemize}
Finally, the EAT must set up and support a corresponding Product Owner
organization through associations of PO's that mirror the SoS's and scale
their PO functions.

\subsection{Outputs/Outcomes of the Scrum Master Cycle}
The SM organization (SoS, SoSoS, and EAT) work as a whole to complete the
other components of the Scrum Master Cycle: \textbf{Continuous Improvement
and Impediment Removal, Cross-Team Coordination, and Deployment}.

The goals of Continuous Improvement and Impediment Removal are to:
\begin{itemize}
\item identify impediments and reframe them as opportunities.
\item maintain a healthy and structured environment for prioritizing and
removing impediments, and then verifying the resulting improvements.
\item ensure visibility in the organization to effect change.
\end{itemize}
The goals of Cross-Team Coordination are to:
\begin{itemize}
\item coordinate similar processes across multiple related teams.
\item mitigate cross-team dependencies to ensure they don't become
impediments.
\item maintain alignment of team norms and guidelines for consistent output.
\end{itemize}

Since the goal of the SoS is to function as a release team, the deployment
of product falls under their scope, while what is contained in any release
falls under the scope of the Product Owners. Therefore, the goals of the
Deployment are to:
\begin{itemize}
\item deliver a consistent flow of valuable finished product to customers.
\item integrate the work of different teams into one seamless product.
\item ensure high quality of the customer experience.
\end{itemize}

\section{Product Owner Cycle}
\subsection{Coordinating the ``What'' - The Product Owner Team}
A group of Product Owners who need to coordinate a shared backlog that
feeds a network of teams are themselves a team called a \textbf{Product Owner Team}.
For each SoS there is an associated Product Owner Team which aligns the
teams' priorities along a single path so that they can coordinate their
team backlogs and build alignment with stakeholders to support the backlog.

A team's product owner is accountable for the composition and prioritization
of the team backlog and may pull backlog items from the shared metascrum backlog
into the team backlog or generate independent backlog items at his or her discretion.

Product Owner Teams hold a scaled version of Backlog Refinement, the \textbf{MetaScrum} 
\begin{itemize}
\item Each team PO (or proxy) must attend the MetaScrum where stakeholders review the Product Owner Team's backlog and address strategy, resources, and date issues.
\item This event is the forum for Leadership, Stakeholders, or other
Customers to express their preferences
\end{itemize}
This event occurs as often as needed, at least once per Sprint, to ensure a
Ready backlog at the Product Owner Team level. 

The main functions of the Product Owner Team are to:
\begin{itemize}
\item create an overarching vision for the product \& make it visible to
the organization.
\item build alignment with key stakeholders to secure support for backlog
implementation.
\item generate a single, prioritized backlog; ensuring that duplication of
work is avoided.
\item assure impediment and technical debt issues are properly prioritized in the backlog.
\item create a minimally uniform ``Definition of Done'' that applies to all teams.
\item resolve dependencies raised by the teams.
\item generate a coordinated Roadmap and Release Plan.
\item decide upon and monitor metrics that give insight into the product and the market.
\end{itemize}
Product Owner Teams, just like SoS's, function as Scrum teams on their own. As such,
they need to have a SM and keeps the team on track in
discussions. They also need a single person who is responsible for coordinating the
generation of a single Product Backlog for all teams within the Scrum of Scrums.
This person is designated as the \textbf{Chief Product Owner}.

\subsection{The Chief Product Owner (CPO)}
The Chief Product Owners coordinate priorities among
Product Owners who work with individual teams. They align backlog
priorities with Stakeholder and Customer needs. Just like a SoSM, they may
be an individual team PO who chooses to play this role as well, or they may
be a person specifically dedicated to this role. Their main
responsibilities are the same as a regular PO's, but at scale:
\begin{itemize}
\item Setting a strategic vision for the whole product.
\item Creating a single, prioritized backlog of value to be delivered by
all of the teams.
\begin{itemize}
\item these items would be larger Product Backlog Items than that for a team PO.
\end{itemize}
\item Working closely with their associated SoSM so that the Release Plan
that the Product Owner team generates can be deployed efficiently.
\item Monitoring customer product feedback and adjusting the backlog
accordingly.
\item Lead the MetaScrum where the Product Backlog is presented and alignment is achieved with stakeholders.
\end{itemize}

\subsection{Scaling the Product Owner Team}
Just as SoS's can grow into SoSoS's, Product Owner Teams can also expand by the same
mechanism. There is no specific term associated with these expanded units,
nor do the CPO's of them have specific expanded titles. We encourage each
organization to develop their own. For the following diagrams, we have
chosen to add an additional ``Chief'' to the title of those PO's as they
magnify out.

The Product Owner Teams meet on a regular cadence (at least once a sprint) with key stakeholders to align the backlog for the organization. This meeting is called a MetaScrum and it is led by the Chief Product Owner. Here, priorities can be negotiated, budgets can be changed, and teams realigned to maximize the deliver of value to the market.

%\pagebreak
Some sample diagrams:

\includegraphics[width=1.0\linewidth]{MetaScrum-R2.png}

\textbf{NOTE:} As mentioned above, these pentagons represent the ideal
sized Scrum teams and ideal sized Product Owner Teams. These diagrams are meant to
be examples only, your organizational diagram may differ greatly.

\subsection{The Executive MetaScrum (EMS)}
The Product Owner Teams enable a network design of Product Owners which is
infinitely scalable alongside their associated SoS's. The Product Owner Team for the
entire agile organization meets with key stakeholders as the \textbf{Executive MetaScrum}. The EMS owns
the organizational vision and sets the strategic priorities, aligning all the teams around common goals.

Sample diagram showing an EMS coordinating 5 groups of 25 teams:

\includegraphics[width=1.0\linewidth]{ExecMetaScrum.png}

\subsection{Outputs/Outcomes of the Product Owner Organization}
The PO organization (the Product Owners, the CPO's, and the Executive
MetaScrum) work as a whole to satisfy the components of the Product Owner
Cycle: \textbf{Strategic Vision, Backlog Prioritization, Backlog
Decomposition \& Refinement, and Release Planning}.

The goals of setting a Strategic Vision are to:
\begin{itemize}
\item clearly align the entire organization along a shared path forward.
\item compellingly articulate why the organization exists.
\item describe what the organization will do to leverage key assets in
support of its mission.
\item respond to rapidly changing market conditions.
\end{itemize}
The goals of Backlog Prioritization are to:
\begin{itemize}
\item identify a clear ordering for products, features, and services to be
delivered.
\item reflect value creation, risk mitigation and internal dependencies in
ordering of the backlog.
\item prioritize the high-level initiatives across the entire agile
organization prior to Backlog Decomposition and Refinement.
\end{itemize}
The goals of Backlog Decomposition \& Refinement are to:
\begin{itemize}
\item break complex products and projects into independent functional
elements that can be completed by one team in one Sprint.
\item capture and distill emerging requirements and customer feedback.
\item ensure all backlog items are truly ``Ready'' so that they can be
pulled by the individual teams.
\end{itemize}
The goals of Release Planning are to:
\begin{itemize}
\item forecast delivery of key features and capabilities.
\item communicate delivery expectations to stakeholders.
\item update prioritization, as needed.
\end{itemize}

Note that Release Planning may encompass one or many releases of the product to a customer.  
It is a higher level planning horizon than a single Sprint, usually covering a period of 1 to 6 months.

\section{Connecting the PO/SM Cycles}

\subsection{Understanding Feedback}
The \textbf{Feedback} component is the second point where the PO \& SM
Cycles touch. Product feedback drives continuous improvement through
adjusting the Product Backlog while Release feedback drives continuous
improvement through adjusting the Deployment mechanisms. The goals of
obtaining and analyzing Feedback are to:
\begin{itemize}
\item validate our assumptions.
\item understand how customers use and interact with the product.
\item capture ideas for new features and functionality.
\item define improvements to existing functionality.
\item update progress towards product/project completion to refine release
planning and stakeholder alignment.
\item identify improvements to deployment methods and mechanisms.
\end{itemize}

\subsection{Metrics \& Transparency}
Radical transparency is essential for Scrum to function optimally, but it
is only possible in an organization that has embraced the Scrum values. It
gives the organization the ability to honestly assess its progress and to
inspect and adapt its products and processes. This is the foundation of the
empirical nature of Scrum as laid out in the Scrum Guide.

Both the SM \& PO Cycles require metrics that will be decided upon by the
separate SM and PO organizations. Metrics may be unique to both specific
organizations as well as to specific functions within those organizations.
Scrum@Scale does not require any specific set of metrics, but it does
suggest that at a bare minimum, the organization should measure:
\begin{itemize}
\item Productivity - e.g. change in amount of Working Product delivered per
Sprint
\item Value Delivery - e.g. business value per unit of team effort
\item Quality - e.g. defect rate or service downtime
\item Sustainability - e.g. team happiness
\end{itemize}
The goals of having Metrics and Transparency are to:
\begin{itemize}
  \item provide all decision makers, including team members, with
appropriate context to make good decisions.
\item shorten feedback cycles as much as possible to avoid over-correction.
\item require minimal additional effort by teams, stakeholders or
leadership.
 \end{itemize}

\subsection{Some notes on Organizational Design}
The scale-free nature of Scrum@Scale allows the design of the organization
to be component-based, just like the framework itself. This permits for
rebalancing or refactoring of teams in response to the market. As an
organization grows, capturing the benefits of distributed teams may be
important. Some organizations reach talent otherwise unavailable and are
able to expand and contract as needed through outsourced development.
Scrum@Scale shows how to do this while avoiding long lag times, compromised
communications, and inferior quality, enabling linear scalability both in
size and global distribution.\footnote{Sutherland, Jeff and Schoonheim,
Guido and Rustenburg, Eelco and Rijk, Maurits, ``Fully distributed scrum:
The secret sauce for hyperproductive offshored development teams'',
AGILE'08. Conference, IEEE: 339-344, 2008}

\includegraphics[width=1.0\linewidth]{VariableSoS-R2.png}
\includegraphics[width=1.0\linewidth]{OrganizationalDiagram.png}

In this organizational diagram, the \textbf{Knowledge \& Infrastructure
Teams} represent virtual teams of specialists of which there are too few to
staff each team. They coordinate with the Scrum teams as a group via
service-level agreements where requests flow through a PO for each
specialty who converts them into a transparent ordered backlog. An
important note is that these teams are NOT silos of individuals who sit
together (this is why they are represented as hollow pentagons); their team
members sit on the actual Scrum teams, but they make up this virtual Scrum
of their own for the purpose of backlog dissemination and process
improvement.

\textbf{Customer Relations, Legal / Compliance, and People Operations} are
included here since they are necessary parts of organizations and will
exist as independent Scrum teams on their own, which all of the others may
rely upon.

A final note on the representation of the EAT \& EMS: in this diagram, they
are shown as overlapping since some members sit on both of the teams. In very
small organizations or implementations, the EAT \& EMS may consist entirely
of the same team members.

\section{End Note}
Scrum@Scale is designed to scale productivity, to get the entire
organization delivering twice the value in half the time with higher quality and
in a significantly improved work environment. Large organizations that
properly implement the framework can cut the cost of their products and
services while improving quality and innovation.

Scrum@Scale is designed to saturate an organization with Scrum. All teams,
including Leadership, Human Resources, Legal, Consulting \& Training, and
product \& service teams, implement the same style of Scrum while
streamlining and enhancing an organization.

Well implemented Scrum can run an entire organization.

\section{Acknowledgements}
We acknowledge IDX for the creation of the Scrum of Scrums which first
allowed Scrum to scale to hundreds of teams,\footnote{Sutherland, Jeff,
``Inventing and Reinventing SCRUM in five Companies'', Sur le site officiel
de l'alliance agile, 2001} PatientKeeper for the creation of the
MetaScrum,\footnote{Sutherland, Jeff, ``Future of scrum: Parallel pipelining
of sprints in complex projects'', Proceedings of the Agile Development
Conference,  IEEE Computer Society 90-102,  2005.} which enabled rapid
deployment of innovative product, and OpenView Venture Partners for scaling
Scrum to the entire organization.\footnote{Sutherland, Jeff and Altman,
Igor, ``Take no prisoners: How a venture capital group does scrum'', Agile
Conference, 2009. AGILE'09, IEEE 350-355.  2009} We value input from Intel
with over 25,000 people doing Scrum who taught us ``nothing scales'' except
a scale-free architecture, and SAP with the largest Scrum team product
organization who taught us management involvement in the MetaScrum is
essential to get 2,000 Scrum teams to work together.

The agile coaches and trainers implementing these concepts at Amazon, GE,
3M, Toyota, Spotify, Maersk, Comcast, AT\&T and many other companies working with Jeff Sutherland
have been helpful in testing these concepts across a wide range of
companies in different domains.

And finally, Avi Schneier and Alex Sutherland have been invaluable in
formulating and editing this document.

\pagebreak

\printbibliography



\end{document}
