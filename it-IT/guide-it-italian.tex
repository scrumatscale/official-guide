\documentclass[12pt,a4paper,parskip=full]{scrartcl}

\usepackage{bbding}
\usepackage{pifont}
\usepackage{wasysym}
\usepackage[margin=1in]{geometry}
\geometry{letterpaper}
\usepackage{xcolor}
\definecolor{red}{HTML}{cc0000}
\definecolor{gray}{HTML}{666666}
\usepackage{sectsty}
\sectionfont{\color{red}}
\subsectionfont{\color{red}}
\usepackage{graphicx}
\usepackage{hyperref}
\usepackage{amssymb}
\usepackage[style=footnote-dw]{biblatex}
\bibliography{S@SGuideBib}
\setlength\bibitemsep{0.5\baselineskip}

\usepackage{enumitem}
\setitemize{noitemsep}
% \setlist{noitemsep, topsep=-5pt}
% \setlength\itemsep{-0.10em}

\renewcommand{\labelitemi}{$\cdot$}
\renewcommand{\labelitemii}{$\cdot$}
\makeatletter
\let\latexl@section\l@section
\def\l@section#1#2{\begingroup\let\numberline\@gobble\latexl@section{#1}{#2}\endgroup}
\makeatother

\usepackage[T1]{fontenc}
\fontfamily{verdana}

\usepackage{scrlayer-scrpage}{}
\makeatletter
\renewcommand{\@seccntformat}[1]{}
\makeatother

\setlength\parindent{0pt}{}

\title{\Huge{\color{red}\textbf{La Guida a Scrum@Scale
\textsuperscript{\copyright}
}}}
\subtitle{\color{gray}La guida definitiva a Scrum@Scale:\\ Lo scaling che funziona}
% \author{}
\date{}


\begin{document}

%\tableofcontents
%\newpage

\section{Scopo della Guida a Scrum@Scale}
Scrum, come originariamente descritto nella Scrum Guide, è un 
framework per sviluppare, consegnare e mantenere prodotti complessi da 
un singolo team. Fin dalla nascita, il suo uso si è esteso alla creazione di 
prodotti, processi, servizi e sistemi che richiedevano il lavoro di più team.
Scrum@Scale è stato creato per coordinare efficientemente questo nuovo 
ecosistema di team in un modo da ottimizzarsi per perseguire la strategia 
globale dell'organizzazione. Viene raggiunto questo obiettivo creando la 
minima burocrazia possibile grazie ad una architettura ad invarianza di scala 
(scale-free architecture), che estende naturalmente il modo in cui i singoli 
team Scrum funzionano lungo tutta l'organizzazione.

Questa guida contiene la definizione dei componenti che costituiscono Scrum@Scale, includendo le corrispondenti su larga scala dei ruoli, degli eventi e degli artefatti, così come le regole che li tengono insieme.

Il Dr. Jeff Sutherland ha sviluppato Scrum@Scale basandosi sui principi fondamentali di Scrum, della teoria dei sistemi adattivi complessi, la teoria dei giochi e la programmazione ad oggetti. Questa guida è stata sviluppata grazie ai suggerimenti di numerosi praticanti esperti di Scrum e basata sui risultati del loro lavoro. L'obiettivo di questa guida è di rendere il lettore in grado di implementare Scrum@Scale in proprio.

\subsection{Perché Scrum@Scale?}
Scrum è stato disegnato per permettere ad un singolo team di lavorare 
alla sua capacità ottimale mantenendo un ritmo sostenibile. Sul campo è
stato notato che all'aumentare dei team scrum all'interno di una organizzazione,
la capacità ottimale (prodotto funzionante) e la velocity di questi team inizia a
decadere (a causa di problemi come le dipendenze tra team e la duplicazione
del lavoro). Diventa quindi ovvio che serviva un framework per coordinare 
efficacemente questi team in modo da ottenere una scalabilità lineare.
Scrum@Scale è stato ideato per raggiungere questo obiettivo grazie alla
sua architettura a invarianza di scala.

Utilizzando una architettura ad invarianza di scala, l'organizzazione non ha 
vincoli di crescere un particolare modo o secondo un insieme arbitrario di
regole; invece può crescere in maniera organica basandosi sui suoi personali
bisogni e ad un ritmo sostenibile di cambiamento, che possa essere 
accettato dai gruppi di individui che compongono l'organizzazione.

Scrum@Scale è pensato per estendersi lungo tutta l'organizzazione: tutti
i dipartimenti, prodotti e servizi. Può essere applicato lungo diversi domini
in qualsiasi tipo di organizzazione in industrie, governi o accademie.

\subsection{Definizione di Scrum@Scale}
Scrum: Un framework con cui le persone possono risolvere problemi adattivi complessi,
producendo e creando creativamente dei prodotti funzionanti del più alto valore possibile.

La Scrum Guide è l'insieme minimo di regole che permettono ispezione e adattabilità
attraverso una trasparenza radicale per creare innovazione, soddisfazione dei clienti,
performance e felicità del team.

Scrum@Scale: Un framework con cui una rete di team Scrum opera in maniera consistente
alla Scrum Guide per indirizzare problemi adattivi complessi producendo e creando creativamente dei prodotti funzionanti del più alto valore possibile.

\textbf{NOTA:} Questi ``prodotti'' possono essere hardware, software, sistemi complessi integrati, processi, servizi, ecc., a seconda del dominio dei team Scrum.

Scrum@Scale è:
\begin{itemize}
\item Leggero - la minima burocrazia sostenibile
\item Semplice da capire - consiste soltanto di team Scrum
\item Difficile da padroneggiare - richiede di implementare un nuovo modello operativo
\end{itemize}

Scrum@Scale è un framework per scalare Scrum. Semplifica radicalmente lo scaling 
in quanto usa Scrum per scalare Scrum.

In Scrum, è stato separata con cura la responsabilità del ``cosa'' dal ``come''.
La stessa cura è stata mantenuta in Scrum@Scale in modo che la giurisdizione e le responsabilità siano espressamente comprese in modo da eliminare conflitti organizzativi e sprechi che impediscano ai team di raggiungere la loro produttività ottimale.

Scrum@Scale è costituito da componenti che consentono ad un'organizzazione di personalizzare la propria strategia di trasformazione e implementazione. Permette di indirizzare le azioni di cambiamento incrementalmente e dando priorità alle aree che si ritiene più importanti, o più bisognose di cambiamento, e di progredire successivamente nelle altre.

Nel separare queste due giurisdizioni, Scrum@Scale contiene due cicli: il ciclo degli Scrum Master (il ``come'') ed il ciclo dei Product Owner (il ``cosa''), intersecandosi in due punti. Presi insieme, questi cicli producono una potente struttura per coordinare gli sforzi di più team lungo un singolo percorso.

\subsection{I componenti del framework Scrum@Scale\textregistered}

\includegraphics[width=1.0\linewidth]{SMPO-Cycle.png}

\subsection{Cultura guidata dai valori}

Oltre a separare la responsabilità del ``cosa'' e del ``come'' Scrum@Scale si propone inoltre di creare organizzazioni sane creando una cultura basata sui valori in un contesto empirico. I valori di Scrum sono: apertura, coraggio, focus, rispetto e impegno. Questi valori guidano un processo decisionale empirico, che dipende dai tre pilastri di trasparenza, ispezione e adattamento.

L'apertura supporta la trasparenza in tutto il lavoro ed i processi, senza la quale non c'è possibilità di ispezionarli onestamente e tentare di adattarli per il meglio. Il coraggio si riferisce a prendere le difficili decisioni richieste dal fornire valore il più rapidamente possibile ed in maniera innovativa.

Focus e impegno si riferiscono al modo con cui gestiamo i nostri obblighi lavorativi, mettendo come più alta priorità la consegna di valore al cliente. Infine, tutto questo deve avvenire in un ambiente basato sul rispetto per gli individui che svolgono il lavoro, senza il quale nulla può essere creato.

Scrum@Scale aiuta le organizzazioni a prosperare supportando un modello di leadership trasformazionale che promuova un ambiente lavorativo positivo con un ritmo sostenibile e mettendo l'impegno a consegnare valore visibile al cliente come frutto dei nostri sforzi.

\subsection{Come iniziare con Scrum@Scale}
Quando si intende implementare una grande rete di team, è fondamentale sviluppare prima un \textbf{Modello di Riferimento} con un piccolo gruppo di team. Qualsiasi carenza si implementi inizialmente, questa verrà poi amplificata quando si estende il metodo a molteplici team. Molti dei problemi iniziali di estensione del metodo saranno di tipo organizzativi, delle procedure e pratiche di sviluppo che bloccano i team da raggiungere prestazioni elevate e creano frustrazione.

Pertanto, la prima sfida è creare un piccolo gruppo di team che implementino bene lo Scrum. Questo insieme di team permette di risolvere gli impedimenti organizzativi che bloccano l'agilità e creare un modello di riferimento di Scrum che funzioni in quel contesto e che può essere utilizzato come riferimento per l'estensione di Scrum in tutta l'organizzazione.

Via via che il modello di riferimento dei team si estende, gli ostacoli ed i colli di bottiglia che ritardano le consegne, producono sprechi o ostacolare l'agilità del business,  diventano visibili. Il modo più efficace per eliminare questi problemi è diffondere Scrum su tutta l'organizzazione in modo che l'intero flusso del valore sia ottimizzato.

Scrum@Scale permette di far scalare linearmente la produttività saturando l'organizzazione con Scrum e distribuendo il carico di lavoro e le responsabilità sulla  qualità organicamente, in maniera consistente con la strategia, i prodotti ed i servizi dell'organizzazione stessa.

\section{Il ciclo degli Scrum Master}
\subsection{Processo a livello di Team}
Il \textbf{Processo a Livello di Team} costituisce il primo punto di contatto tra lo Scrum Master ed il ciclo dei Product Owner, ed è definito chiaramente nella Scrum Guide. È composto da tre artefatti, cinque eventi e tre ruoli. L'obbiettivo del processo a livello di Team è di:
\begin{itemize}
\item massimizzare il flusso di lavoro completato e verificato qualitativamente.
\item incrementare le performance del team nel corso del tempo.
\item operare in una maniera che sostenibile e di arricchimento per il Team.
\end{itemize}

\subsection{Coordinating the ``How'' - The Scrum of Scrums}
A set of the teams that have a need to coordinate comprise a \textbf{``Scrum of Scrums'' (SoS)}. This team is a Scrum Team unto itself which is responsible for a fully integrated set of potentially shippable increments of product at the end of every Sprint from all participating teams. A SoS functions as a Release Team and must be able to directly deliver value to customers. To do so effectively, it needs to be consistent with the Scrum Guide; that is, have its own roles, artifacts, and events:

Roles:

It needs to have all of the skills necessary to deliver a fully integrated potentially shippable product at the end of every Sprint. (It may need experienced architects, QA Leaders, and other operational skill sets.) Since the SoS needs to be responsive in real-time to impediments raised by participating teams, the Scrum Masters of the participating teams need to be on the Scrum of Scrums.
It has Product Owner representation to resolve prioritization issues.
The Scrum Master of the Scrum of Scrums is called the \textbf{Scrum of Scrums Scrum Master (SoSM)}.

Events:

A Backlog Refinement event wherein they decide what impediments are ``ready'' to be removed, how best to remove them, and how the team will know it is ``done.''
Particular attention should be paid to the SoS Retrospective in which the teams' representatives share any learnings or process improvements that their individual teams have succeeded with, in order to standardize those practices across the teams within the SoS.
The SoS holds a \textbf{Scaled Daily Scrum (SDS)}. The SDS event mirrors the Daily Scrum in that it optimizes the collaboration and performance of the network of teams. Since the SoS needs to be responsive in real-time to impediments raised by participating teams, the Scrum Masters of the participating teams need to be at the Scaled Daily Scrum. It should also have Product Owner representation to resolve prioritization issues. Any person or number of people from participating teams may attend as needed.

Additionally, the SDS:

\begin{itemize}
\item is time-boxed to 15 minutes or less.
\item must be attended by a representative of each team including the Product Owner team.
\item is a forum where team representatives discuss what is going well, what is getting done, and how teams can work together more effectively. Some examples of what might be discussed are:
\begin{itemize}
\item What impediments does my team have that will prevent them from
accomplishing their Sprint Goal (or impact the upcoming release)?
\item Is my team doing anything that will prevent another team from
accomplishing their Sprint Goal (or impact their upcoming release)?
\item Have we discovered any new dependencies between the teams or
discovered a way to resolve an existing dependency?
\item What improvements have we discovered that can be leveraged across teams?
\end{itemize}
\end{itemize}

\subsection{The Scrum of Scrums Scrum Master (SoSM)}
The Scrum of Scrums Scrum Master (SoSM) is accountable for the release of the
joint teams' effort and must:
\begin{itemize}
\item make progress visible.
\item make an impediment backlog visible to the organization.
\item remove impediments that the teams cannot address themselves.
\item facilitate prioritization of impediments, with particular attention to cross-team
dependencies and the distribution of backlog.
\item improve the efficacy of the Scrum of Scrums.
\item work closely with the Product Owners to deploy a potentially
releasable Product Increment at least every Sprint.
\item coordinate the teams' deployment with the Product Owner's Release
Plans.
\end{itemize}

\subsection{Scaling the SoS}
Depending upon the size of the organization or implementation, more than
one SoS may be needed to deliver a very complex product. In those cases, a
\textbf{Scrum of Scrum of Scrums (SoSoS)} can be created out of multiple
Scrums of Scrums. The SoSoS is an organic pattern of Scrum teams which is
infinitely scalable. Each SoSoS should have SoSoSM's and scaled versions of
each artifact \& event.

Scaling the SoS reduces the number of communication pathways within the
organization so that complexity is encapsulated. The SoSoS interfaces with
a SoS in the exact same manner that a SoS interfaces with a single Scrum
team which allows for linear scalability.

\pagebreak
Sample Diagrams:

\includegraphics[width=1.0\linewidth]{Sos-R2.png}

\textbf{\textsc{note:}} While the Scrum Guide defines the optimal team size as being
3 to 9 people, Harvard research determined that optimal team size is 4.6
people.\footnote{Hackman, J Richard, Leading teams: Setting the stage for
great performances, Harvard Business Press, 2002} Experiments with high
performing Scrum teams have repeatedly shown that 4 or 5 people doing the
work is the optimal size. It is essential to linear scalability that this
pattern be the same for the number of teams in a SoS. Therefore, in the
above and following diagrams, pentagons were chosen to represent a team of
5. These diagrams are meant to be examples only, your organizational
diagram may differ greatly.

\subsection{The Executive Action Team}
The Scrum of Scrums for the entire agile organization is called the
\textbf{Executive Action Team (EAT)}. The leadership team creates an agile bubble
in the organization where the reference model operates with 
its own guidelines and procedures that integrates effectively 
with any part of the organization that is not agile. It owns the agile ecosystem, 
implements the Scrum values, and assures that 
Scrum roles are created and supported.

The EAT is the final stop for
impediments that cannot be removed by the SoS's that feed it. Therefore, it
must be comprised of individuals who are empowered, politically and
financially, to remove them. 
The function of the EAT is to coordinate
multiple SoS's (or SoSoS's). As with any Scrum team, it needs a PO and SM.
It would be best if the EAT met daily as a Scrum team. They must meet at
least once per Sprint and have a transparent backlog.

%\pagebreak
Sample Diagram showing an EAT coordinating 5 groupings of 25 teams:

\includegraphics[width=\textwidth,height=\textheight,keepaspectratio]{SoS-EAT.png}

\subsection{The EAT's Backlog \& Responsibilities}
Scrum is an agile operating system that is different from traditional
project management. The entire SM organization reports into the EAT, which
is responsible for implementing this agile operating system by
establishing, maintaining, and enhancing the implementation in the
organization.
The EAT's role is to create an Organizational Transformation Backlog (a
prioritized list of the agile initiatives that need to be accomplished) and
see that it is carried out. For example, if there is a traditional Product
Development Life Cycle in the old organization, a new agile Product
Development Life Cycle needs to be created, implemented, and supported. It
will typically support quality and compliance issues better than the old
method but be implemented in a different way with different rules and
guidelines. The EAT ensures that a Product Owner organization is created and funded
and that this organization is represented on the EAT to support these efforts.

The EAT is accountable for the quality of Scrum within the organization.
Its responsibilities include but are not limited to:
\begin{itemize}
\item creating an agile operating system for the Reference Model as it
scales through the organization, including corporate operational rules,
procedures, and guidelines to enable agility.
\item measuring and improving the quality of Scrum in the organization.
\item building capability within the organization for business agility.
\item creating a center for continuous learning for Scrum professionals.
\item supporting the exploration of new ways of working.
\end{itemize}
Finally, the EAT must set up and support a corresponding Product Owner
organization through associations of PO's that mirror the SoS's and scale
their PO functions. These teams of PO's and key stakeholders are known as
\textbf{MetaScrums}.

\subsection{Outputs/Outcomes of the Scrum Master Cycle}
The SM organization (SoS, SoSoS, and EAT) work as a whole to complete the
other components of the Scrum Master Cycle: \textbf{Continuous Improvement
and Impediment Removal, Cross-Team Coordination, and Deployment}.

The goals of Continuous Improvement and Impediment Removal are to:
\begin{itemize}
\item identify impediments and reframe them as opportunities.
\item maintain a healthy and structured environment for prioritizing and
removing impediments, and then verifying the resulting improvements.
\item ensure visibility in the organization to effect change.
\end{itemize}
The goals of Cross-Team Coordination are to:
\begin{itemize}
\item coordinate similar processes across multiple related teams.
\item mitigate cross-team dependencies to ensure they don't become
impediments.
\item maintain alignment of team norms and guidelines for consistent output.
\end{itemize}

Since the goal of the SoS is to function as a release team, the deployment
of product falls under their scope, while what is contained in any release
falls under the scope of the Product Owners. Therefore, the goals of the
Deployment are to:
\begin{itemize}
\item deliver a consistent flow of valuable finished product to customers.
\item integrate the work of different teams into one seamless product.
\item ensure high quality of the customer experience.
\end{itemize}

\section{Product Owner Cycle}
\subsection{Coordinating the ``What'' - The MetaScrum}
A group of Product Owners who need to coordinate a single backlog that
feeds a network of teams are themselves a team called a \textbf{MetaScrum}.
For each SoS there is an associated MetaScrum. A MetaScrum aligns the
teams' priorities along a single path so that they can coordinate their
backlogs and build alignment with stakeholders to support the backlog.

MetaScrums hold a scaled version of Backlog Refinement, the \textbf{Scaled Backlog Refinement Meeting} 
\begin{itemize}
\item Each team PO (or proxy) must attend
\item This event is the forum for Leadership, Stakeholders, or other
Customers to express their preferences
\end{itemize}
This event occurs as often as needed, at least once per Sprint, to ensure a
Ready backlog. 

The main functions of the MetaScrum are to:
\begin{itemize}
\item create an overarching vision for the product \& make it visible to
the organization.
\item build alignment with key stakeholders to secure support for backlog
implementation.
\item generate a single, prioritized backlog; ensuring that duplication of
work is avoided.
\item create a uniform ``Definition of Done'' that applies to all teams in
the SoS.
\item eliminate dependencies raised by the SoS.
\item generate a coordinated Release Plan.
\item decide upon and monitor metrics that give insight into the product.
\end{itemize}
MetaScrums, just like SoS's, function as Scrum teams on their own. As such,
they need to have someone who acts as a SM and keeps the team on track in
discussions. They also need a single person who is responsible for coordinating the
generation of a single Product Backlog for all of the teams covered by the
MetaScrum. This person is designated as the \textbf{Chief Product Owner}.

\subsection{The Chief Product Owner (CPO)}
Through the MetaScrums, Chief Product Owners coordinate priorities among
Product Owners who work with individual teams. They align backlog
priorities with Stakeholder and Customer needs. Just like a SoSM, they may
be an individual team PO who chooses to play this role as well, or they may
be a person specifically dedicated to this role. Their main
responsibilities are the same as a regular PO's, but at scale:
\begin{itemize}
\item Setting a strategic vision for the whole product.
\item Creating a single, prioritized backlog of value to be delivered by
all of the teams.
\begin{itemize}
\item These items would be larger stories than that for a team PO.
\end{itemize}
\item Working closely with their associated SoSM so that the Release Plan
that the MetaScrum team generates can be deployed efficiently.
\item Monitoring customer product feedback and adjusting the backlog
accordingly.
\end{itemize}

\subsection{Scaling the MetaScrum}
Just as SoS's can grow into SoSoS's, MetaScrums can also expand by the same
mechanism. There is no specific term associated with these expanded units,
nor do the CPO's of them have specific expanded titles. We encourage each
organization to develop their own. For the following diagrams, we have
chosen to add an additional ``Chief'' to the title of those PO's as they
magnify out.

%\pagebreak
Some sample diagrams:

\includegraphics[width=1.0\linewidth]{MetaScrum-R2.png}

\textbf{NOTE:} As mentioned above, these pentagons represent the ideal
sized Scrum teams and ideal sized MetaScrums. These diagrams are meant to
be examples only, your organizational diagram may differ greatly.

\subsection{The Executive MetaScrum (EMS)}
The MetaScrums enable a network design of Product Owners which is
infinitely scalable alongside their associated SoS's. The MetaScrum for the
entire agile organization is the \textbf{Executive MetaScrum}. The EMS owns
the organizational vision and sets the strategic priorities for the whole
company, aligning all the teams around common goals.

Sample diagram showing an EMS coordinating 5 groups of 25 teams:

\includegraphics[width=1.0\linewidth]{ExecMetaScrum.png}

\subsection{Outputs/Outcomes of the Product Owner Organization}
The PO organization (various MetaScrums, the CPO's, and the Executive
MetaScrum) work as a whole to satisfy the components of the Product Owner
Cycle: \textbf{Strategic Vision, Backlog Prioritization, Backlog
Decomposition \& Refinement, and Release Planning}.

The goals of setting a Strategic Vision are to:
\begin{itemize}
\item clearly align the entire organization along a shared path forward.
\item compellingly articulate why the organization exists.
\item describe what the organization will do to leverage key assets in
support of its mission.
\item respond to rapidly changing market conditions.
\end{itemize}
The goals of Backlog Prioritization are to:
\begin{itemize}
\item identify a clear ordering for products, features, and services to be
delivered.
\item reflect value creation, risk mitigation and internal dependencies in
ordering of the backlog.
\item prioritize the high-level initiatives across the entire agile
organization prior to Backlog Decomposition and Refinement.
\end{itemize}
The goals of Backlog Decomposition \& Refinement are to:
\begin{itemize}
\item break complex products and projects into independent functional
elements that can be completed by one team in one Sprint.
\item capture and distill emerging requirements and customer feedback.
\item ensure all backlog items are truly ``Ready'' so that they can be
pulled by the individual teams.
\end{itemize}
The goals of Release Planning are to:
\begin{itemize}
\item forecast delivery of key features and capabilities.
\item communicate delivery expectations to stakeholders.
\item update prioritization, as needed.
\end{itemize}

\section{Connecting the PO/SM Cycles}

\subsection{Understanding Feedback}
The \textbf{Feedback} component is the second point where the PO \& SM
Cycles touch. Product feedback drives continuous improvement through
adjusting the Product Backlog while Release feedback drives continuous
improvement through adjusting the Deployment mechanisms. The goals of
obtaining and analyzing Feedback are to:
\begin{itemize}
\item validate our assumptions.
\item understand how customers use and interact with the product.
\item capture ideas for new features and functionality.
\item define improvements to existing functionality.
\item update progress towards product/project completion to refine release
planning and stakeholder alignment.
\item identify improvements to deployment methods and mechanisms.
\end{itemize}

\subsection{Metrics \& Transparency}
Radical transparency is essential for Scrum to function optimally, but it
is only possible in an organization that has embraced the Scrum values. It
gives the organization the ability to honestly assess its progress and to
inspect and adapt its products and processes. This is the foundation of the
empirical nature of Scrum as laid out in the Scrum Guide.

Both the SM \& PO Cycles require metrics that will be decided upon by the
separate SM and PO organizations. Metrics may be unique to both specific
organizations as well as to specific functions within those organizations.
Scrum@Scale does not require any specific set of metrics, but it does
suggest that at a bare minimum, the organization should measure:
\begin{itemize}
\item Productivity - e.g. change in amount of Working Product delivered per
Sprint
\item Value Delivery - e.g. business value per unit of team effort
\item Quality - e.g. defect rate or service downtime
\item Sustainability - e.g. team happiness
\end{itemize}
The goals of having Metrics and Transparency are to:
\begin{itemize}
  \item provide all decision makers, including team members, with
appropriate context to make good decisions.
\item shorten feedback cycles as much as possible to avoid over-correction.
\item require minimal additional effort by teams, stakeholders or
leadership.
 \end{itemize}

\subsection{Some notes on Organizational Design}
The scale-free nature of Scrum@Scale allows the design of the organization
to be component-based, just like the framework itself. This permits for
rebalancing or refactoring of teams in response to the market. As an
organization grows, capturing the benefits of distributed teams may be
important. Some organizations reach talent otherwise unavailable and are
able to expand and contract as needed through outsourced development.
Scrum@Scale shows how to do this while avoiding long lag times, compromised
communications, and inferior quality, enabling linear scalability both in
size and global distribution.\footnote{Sutherland, Jeff and Schoonheim,
Guido and Rustenburg, Eelco and Rijk, Maurits, ``Fully distributed scrum:
The secret sauce for hyperproductive offshored development teams'',
AGILE'08. Conference, IEEE: 339-344, 2008}

\includegraphics[width=1.0\linewidth]{VariableSoS-R2.png}
\includegraphics[width=1.0\linewidth]{OrganizationalDiagram.png}

In this organizational diagram, the \textbf{Knowledge \& Infrastructure
Teams} represent virtual teams of specialists of which there are too few to
staff each team. They coordinate with the Scrum teams as a group via
service-level agreements where requests flow through a PO for each
specialty who converts them into a transparent ordered backlog. An
important note is that these teams are NOT silos of individuals who sit
together (this is why they are represented as hollow pentagons); their team
members sit on the actual Scrum teams, but they make up this virtual Scrum
of their own for the purpose of backlog dissemination and process
improvement.

\textbf{Customer Relations, Legal / Compliance, and People Operations} are
included here since they are necessary parts of organizations and will
exist as independent Scrum teams on their own, which all of the others may
rely upon.

A final note on the representation of the EAT \& EMS: in this diagram, they
are shown as overlapping since some members sit on both of the teams. In very
small organizations or implementations, the EAT \& EMS may consist entirely
of the same team members.

\section{End Note}
Scrum@Scale is designed to scale productivity, to get the entire
organization doing twice the work in half the time with higher quality and
in a significantly improved work environment. Large organizations that
properly implement the framework can cut the cost of their products and
services while improving quality and innovation.

Scrum@Scale is designed to saturate an organization with Scrum. All teams,
including Leadership, Human Resources, Legal, Consulting \& Training, and
product \& service teams, implement the same style of Scrum while
streamlining and enhancing an organization.

Well implemented Scrum can run an entire organization.

\section{Acknowledgements}
We acknowledge IDX for the creation of the Scrum of Scrums which first
allowed Scrum to scale to hundreds of teams,\footnote{Sutherland, Jeff,
``Inventing and Reinventing SCRUM in five Companies'', Sur le site officiel
de l'alliance agile, 2001} PatientKeeper for the creation of the
MetaScrum,\footnote{Sutherland, Jeff, ``Future of scrum: Parallel pipelining
of sprints in complex projects'', Proceedings of the Agile Development
Conference,  IEEE Computer Society 90-102,  2005.} which enabled rapid
deployment of innovative product, and OpenView Venture Partners for scaling
Scrum to the entire organization.\footnote{Sutherland, Jeff and Altman,
Igor, ``Take no prisoners: How a venture capital group does scrum'', Agile
Conference, 2009. AGILE'09, IEEE 350-355.  2009} We value input from Intel
with over 25,000 people doing Scrum who taught us ``nothing scales'' except
a scale-free architecture, and SAP with the largest Scrum team product
organization who taught us management involvement in the MetaScrum is
essential to get 2,000 Scrum teams to work together.

The agile coaches and trainers implementing these concepts at Amazon, GE,
3M, Toyota, Spotify, Maersk, Comcast, AT\&T and many other companies working with Jeff Sutherland
have been helpful in testing these concepts across a wide range of
companies in different domains.

And finally, Avi Schneier and Alex Sutherland have been invaluable in
formulating and editing this document.

\pagebreak

\printbibliography



\end{document}
